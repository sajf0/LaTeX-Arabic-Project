\chapter{مفاهيم أساسية}

\newpage

\section{تعريف الزمرة}

\begin{definition}
	
	يكون النظام الرياضي \((G, *)\) زمرة حيث \(G\) مجموعة غير خالية و \(*\) عملية ثنائية على \(G\) إذا تحقق ما يلي
	
	\begin{enumerate}
		
		\item الإنغلاق: \(\forall a, b \in G \Rightarrow a * b \in G\)
		
		\item التجميع: \(\forall a, b, c \in G\, (a * (b * c) = (a * b) * c)\)
	\end{enumerate}
	
	
	\begin{theorem}
		
		إذا كانت \((G, *)\) زمرة فإنه يوجد عنصر محايد وحيد و لكل عنصر يوجد نظير وجيد 
		
	\end{theorem}
	
	\begin{myproof}
		نفرض أن \(e, e'\) عنصرين محاييدين في \((G, *)\) \\
		بما أن \(e'\) عنصر محايد فإن \(a * e = e * a = a\) لكل \(a\in G\) و منه \(e * e' = e'\)
		\begin{gather*}
			\Rightarrow a * b = e \wedge a * c = e \Rightarrow a * b = a * c \\
			\Rightarrow b * (a * b) = b * (a * c) \Rightarrow (b * a) * b = (b * a) * c
		\end{gather*}
	\end{myproof}
	
\end{definition}